\documentclass[10pt,a4paper]{letter}
\usepackage[utf8]{inputenc}
\usepackage{amsmath}
\usepackage{amsfonts}
\usepackage{amssymb}
\address{Mohammed Aït Lahcen \\ Friedensgasse 69 \\ 4056 Basel, Switzerland} 
\signature{Mohammed Aït Lahcen} 
\begin{document} 
\begin{letter}{Federal Reserve Bank of St. Louis \\ P.O. Box 442 \\ St. Louis, MO 63166-0442} 
\opening{Dear Madam or Sir,} 
 
I appreciate this opportunity to provide further details in support of my application for a dissertation internship at the Federal Reserve Bank of St. Louis.

I am currently a 3rd year PhD student enrolled in the PhD Program in Applied Economics at the University of Basel where I work under the supervision of Professor Dr. Aleksander Berentsen, a Research Fellow of your institution. Last year, I finished successfully my coursework requirements within the Swiss Doctoral Program in Economics at the Study Center Gerzensee. I had also the opportunity of attending several advanced courses on monetary economics, labor macroeconomics and computational economics.

My current research is focused on monetary economics for developing countries. I am chiefly interested in applying and adapting modern monetary theory and models to emerging and developing economies. During my Master studies, I visited the Central Bank of Morocco to develop and estimate a New-Keynesian DSGE model for the Moroccan economy. The model was particularly tailored to the specifics of a small open developing economy including an informal sector, a dual labor market as well as some degree of capital controls.

The first chapter of my PhD dissertation consists in developing a search and matching framework that captures the long-run relationship between inflation and unemployment in the presence of informality in both labor and goods markets. The resulting theoretical model is based on the works of Mortensen and Pissarides (1994) and Lagos and Wright (2005) and integrates both the informal production sector and the informal employment into a unified framework that focuses on the use of money in informal transactions. I calibrate the model to the Brazilian economy and provide useful numerical results which point to the importance of accounting for informality when considering the inflation-unemployment trade-off in the conduct of monetary policy in developing countries.

During the dissertation internship I intend to work on the second chapter of my dissertation which is a joint project with Dr. Pedro Gomis-Porqueras, Professor of Economics at Deakin University, Australia. This project addresses the issue of financial inclusion in a micro-founded model of money and banking based on the works of Lagos and Wright (2005) and Berentsen, Camera and Waller (2007). Our objective is to develop a theory that sheds some light on the effects of monetary policy on banking competition and the access of consumers to basic financial services in economies with low levels of financial inclusion.

I had the pleasure of meeting the head of the research division as well as some economists working at your institution during a few conferences and was impressed by their level of expertise and the depth of their economic knowledge. A dissertation internship at the Federal Reserve Bank of St. Louis will offer me the opportunity to expose my work to such brilliant minds and gain valuable insights and comments to further improve my research work which I hope will prove useful to policy makers around the world and in particular to central bankers in emerging and developing countries. 

For all that has been stated above, I assure you that it is my desire and will be a true pleasure and a unique opportunity for me to be accepted for a dissertation internship at your institution.

Thank you in advance for the time and the attention you will grant to my application.

 
\closing{Best regards,} 
%\cc{Cclist} 
%\ps{adding a postscript} 
%\encl{Curriculum Vitae \\ Dissertation Proposal \\ Reference Letter} 
\end{letter} 
\end{document}