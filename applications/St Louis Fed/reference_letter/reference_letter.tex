\documentclass[10pt,a4paper]{article}
\usepackage[utf8]{inputenc}
\usepackage{amsmath}
\usepackage{amsfonts}
\usepackage{amssymb}
%\author{Mohammed Ait Lahcen}
\title{Letter of Recommendation}
\date{}
\begin{document}

\maketitle

It is a true pleasure for me to write this letter in recommendation of my student Mohammed Aït Lahcen for the Zurich Initiative on Computational Economics (ZICE) 2016. I know Mohammed Aït Lahcen in my role as his PhD thesis advisor within the PhD program for Applied Economics at the university of Basel.

I first met Mohammed a year ago when he came to interview for the PhD program in Basel. I interviewed him and was satisfied by his motivation... 

Mohammed holds a Master of Science degree in Economics from the University of Lausanne and a Bachelor degree in Finance and Accounting from ISCAE Business School, Casablanca, Morocco. He had a first experience in the finance industry before pursuing a PhD degree.

As part of our PhD program, Mohammed is currently enrolled in the Swiss Doctoral Program in Economics at the Study Center Gerzensee. This program provides core training in macroeconomics, microeconomics and econometrics in line with what is offered in the top US universities.

Last term, Mohammed was the teaching assistant for our Master's course in Monetary Economics. He was responsible of offering tutorial sessions, holding office hours, attending and grading students' presentations and writing and correcting the final exam. During his TA work, he demonstrated an ability to work independently with great enthusiasm and performed his duties in a diligent and professional manner to the students satisfaction.

For his Master thesis, Mohammed Aït Lahcen did an excellent work on introducing informality into a small open economy New-Keynesian DSGE model to explain the shock absorbing properties of the informal economy. His current research is oriented towards development macroeconomics. He works on building a theoretical framework to understand the relationship between inflation and unemployment in the presence of a sizeable informal economy. To do that, he plans to integrate both the informal production sector and the informal employment into a search theoretic framework by focusing on tax evasion as the main defining criterion of informality. I strongly believe this line of research is very promising and that it will offer a better understanding of the interactions between the formal and informal sectors in developing economies.

I strongly believe that attending the ZICE 2016 wo1rkshop will be of great benefit to Mohammed Aït Lahcen in helping him master the necessary numerical skills and tools he will need in the pursuit of his research agenda.

I would be happy to provide any further details in support of this recommendation if requested.
\\


Sincerely,



\end{document}