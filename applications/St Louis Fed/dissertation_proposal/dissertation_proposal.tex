\documentclass[12pt,a4paper,titlepage]{article}
\usepackage{amsmath}
\usepackage{amssymb}
\usepackage[utf8]{inputenc}
\usepackage{csquotes}
\usepackage[onehalfspacing]{setspace}
\usepackage{geometry}
\geometry{a4paper, top=25mm, left=25mm, right=25mm, bottom=30mm,
headsep=10mm, footskip=12mm}
\usepackage[english]{babel}
\usepackage{graphicx}
\usepackage[round, authoryear, sort&compress]{natbib} % Use the natbib reference package - read up on this to edit the reference style; if you want text (e.g. Smith et al., 2012) for the in-text references (instead of numbers), remove 'numbers' 
\usepackage{setspace}
\usepackage{hyperref}


\title{Research Proposal}

\author{Mohammed Aït Lahcen \\
University of Basel}

\date{\today}

\begin{document}
%\maketitle


\begin{titlepage}


\begin{center}
\vspace*{10em}
\large \vspace{1em}
\huge Dissertation Proposal \\
\vspace{.5em}
\large Mohammed Aït Lahcen\\
\small Advisor: Prof. Dr. Aleksander Berentsen\\
\vspace{2em}
Faculty of Business and Economics\\
University of Basel\\
\vspace{4em}
\large

\end{center}


\end{titlepage}

%\begin{abstract}
%Your abstract.
%\end{abstract}

\pagebreak



\pagebreak

\section{Introduction}

This dissertation is part of the growing literature on development macroeconomics in general and the literature on the monetary economics of developing countries in particular. Broadly described, development macroeconomics is concerned with applying and adapting modern macroeconomic theory and models to emerging and developing economies. Indeed, most of modern macroeconomics was conceived with developed economies in mind and the resulting theories were tested mostly against data from developed countries, in particular the US. However, developing economies tend to differ from developed ones in some important aspects such as the existence of a large informal sector, vulnerability to external shocks and a weak level of financial development. Trying to apply the same models created for developed economies to the study of developing economies without taking into account their economic and institutional idiosyncrasies might result in wrong conclusions and lead to counter-productive policy recommendations.

The literature on development macroeconomics can be traced back to the debate between monetarists and structuralists about the sources of inflation in Latin American developing economies during the 1960s \citep{Agenor2008}. Throughout the crises that hit developing countries starting from the 1970s, development macroeconomists turned to other issues such as exchange rate regimes, capital controls, informality, financial development and inclusion and labor market institutions.

In my dissertation I intend to address some theoretical and empirical elements which I believe are essential in understanding the monetary economics of developing countries but are not necessarily relevant when analyzing most developed economies. In particular, I focus on two aspects of developing economies namely informality and financial inclusion and their relationship to money, inflation and monetary policy. In what follows I present more details about each project, the progress that I made so far and the steps I intend to follow towards the completion of my dissertation.


\section{Money, Inflation, and Unemployment in the Presence of Informality}

(A working paper version of this project is available \href{http://www.econ.uzh.ch/static/workingpapers.php?id=927}{here}).
\\

The long-run relationship between inflation and unemployment in developed economies has been studied extensively. Several authors found compelling evidence against a vertical long-run Philips curve \citep{Karanassou2003, Beyer2007, Schreiber2007, Berentsen2011} which stands at odds with the assumption of long-run neutrality of money and makes this issue again relevant for welfare and policy analysis. Despite this evidence, the literature on the shape of the long-run Philips curve (LRPC, henceforth) in developing economies is almost non-existent. It is not clear why this issue has been ignored by development macroeconomists especially with regard to the high inflation rates observed in many developing countries. Furthermore, the relationship between inflation and unemployment in the context of a developing economy might be affected by specific institutional failures not present in developed economies such as strong informality, weak labor institutions and limited financial inclusion. This work aims to help fill this gap in the literature by studying the impact of informality on the shape of the LRPC.

There is a strong disagreement in the academic literature and among international institutions about the definition of informality. Some of the main ongoing issues concern both the relationship between legal and illegal informal activities and the distinction between informality and the household sector \citep{ILO2013}. In this paper, I define informality as the market-based production of legal goods and services which is unreported to the government. This definition excludes the household sector as well as criminal activities. The deliberate effort to conceal informal activities from the government is an important aspect in the distinction between the informal sector and the household sector which encompasses all domestically produced goods and services that are exempt from taxation and labor regulation. Furthermore, the inherent legal nature of informal activities stands in contrast to the illegal nature of criminal activities even if both are characterized by the active avoidance of government monitoring.

The presence of a sizable informal sector within developing economies has been one of the main issues studied by development macroeconomists. Within the literature on informality, one can distinguish between two main currents. The first one is empirical and is chiefly concerned with the measurement of informality using reduced-form econometric models. This approach faces considerable methodological challenges to the extent that agents operating within the informal economy are relentlessly trying to conceal any traces of their activities. To overcome these challenges, economists resort to indirect measurement methods using electricity consumption, money demand and cash transactions, mismatches in national accounts or household surveys. \cite{Schneider2011} provide an extensive survey of this literature.

The second current proposes theoretical models capable of replicating some aspects of agents behavior in order to provide policy recommendations. Early work on this literature focused on the dualistic view of the economy by modeling the presence of an informal sector as the result of barriers to entry into the formal sector facing firms and workers. In more recent models, the informal sector comes to existence as an optimal response to regulatory pressure such as high income tax or stringent labor regulations.

Within the second current, there exists a growing strand studying the effect of different labor institutions and policies on the labor market in the presence of informal employment \citep{Fugazza2004, Zenou2008, Albrecht2009, Ulyssea2010, Bosch2012, Bosch2014, Charlot2015, Meghir2015}. Another strand focuses on explaining the allocation choices of workers and firms between formal and informal jobs \citep{Garcia2013, Almeida2012, Boeri2005}. This literature also covers other aspects of informal employment such as cyclical flows and transition dynamics between formal and informal labor markets \citep{Bosch2008, Bosch2010}, the wage differentials between formal and informal jobs \citep{ElBadaoui2010} and informal self-employment \citep{Narita2014, Albrecht2009, Satchi2009}.

Several authors studied the relationship between taxation and informality \citep{Gomis-Porqueras2008, Prado2011, Ihrig2004, Aruoba2010}. These authors usually assume that entering the informal economy is mainly the result of a decision to avoid taxes. The trade-off is often between paying taxes and being part of the formal economy or avoiding taxes and paying a penalty in case of detection by the fiscal authority. The degree of monitoring, legal enforcement and punishment against informality play a strong role in this trade-off. Those parameters can be proxies for the rule of law and the presence of strong institutions which provides broader explanations for the size of the informal sector. One recurrent finding is that the size of the informal sector is negatively affected by an increased level of enforcement or a reduction in the tax rate while weak institutions are often associated with high levels of informality.

This paper stands close to the informality literature dealing with money and inflation. Most of the empirical evidence in this literature points towards a positive relationship between inflation and the size of the informal economy on a cross-country level. Some authors explain this by the use of seigniorage as a source of income for the government in some developing countries due to low tax revenues resulting from a large informal sector. For example, \cite{Koreshkova2006} develops a cash-in-advance model that relates the size of the informal sector to the trade-off between inflation and the ensuing share of seigniorage in the government revenues in one hand and the income tax in the other. She shows that inflation works to smooth the tax burden between the formal and informal parts of the economy. This result offers an interesting explanation for the very high inflation rates observed in some poor countries. \cite{Gomis-Porqueras2014} introduce a money search model where agents can avoid taxes on part of their income by using cash transactions. They derive a model-based measure of informality and produce country-level estimates which tend to be on the lower range of the more traditional reduced-form estimates such as those reported by \cite{Enste2000}. This can be explained by the stronger restrictions their theoretical model imposes on data as well as the exclusion of illegal activities from their measurements. \cite{Bittencourt2014} use a monetary overlapping generations model with endogenous tax evasion to study the effect of both financial sector development and inflation on the size of the informal economy. They find that a lower level of financial development provides agents with a higher incentive in participating in tax evasion activities.

The contribution of this paper consists in developing a theoretical model that captures the relationship between inflation and unemployment in the presence of informality in both labor and goods markets. To this purpose, I introduce a dual labor market into a dynamic general equilibrium model with search frictions similar to \cite{Berentsen2011}. I extend the model of \cite{Berentsen2011} mainly in three directions. First, I add stochastic job creation which allows for an endogenously determined size of formal and informal employment. Second, I add credit as a means of payment in formal transactions alongside money. Third, I create a link between the ability of firms to offer credit in trading transactions and their decision to hire workers formally or informally. These three ingredients allow for interactions between inflation, unemployment and informality.

In the model, inflation affects unemployment through two channels: the matching channel and the hiring channel. On one hand, higher inflation reduces the surplus of monetary trades thus lowering firms entry and increasing unemployment. On the other hand, the lower impact of inflation on formal transactions where credit is partially available shifts the hiring decision towards formal jobs with a lower separation rate thus reducing unemployment. The matching technology in the goods market provides an amplification mechanism which can strengthen both channels. The equilibrium effect is ambiguous and depends on the size of the informal sector and the availability of credit in formal transactions. 

I calibrate the model to match some long-run statistics of the Brazilian economy. The reason I focus my numerical analysis on Brazil is that it has a sizable informal sector, close to the average level in Latin American countries \citep{Perry2007}, and relatively good statistics about it. This is because the distinction in Brazil between formal and informal work is quite straightforward \citep{Gerard2016}.

The calibration exercise focuses on the long-run properties of the data by matching certain model-based statistics with their equivalent long-run empirical averages. I separate the model's parameters into two groups: exogenous parameters and jointly calibrated parameters. The first group consists of parameters which I directly set to a specific value. Once all the exogenous parameters are set I jointly calibrate the second group of parameters with the equilibrium solution such that the model matches the empirical targets. More specifically, the joint calibration procedure consists in solving for the vector of calibrated parameters and the equilibrium solution which together reduce the distance (squared percentage difference) between the targeted statistics and the corresponding model statistics while taking the system of steady state equilibrium equations as equality constraints and imposing interval bounds on the value of some parameters.

In the current version of the paper, I report elastcities of various measures to changes in some policy parameters. The model is able to replicate most of the stylized facts found in the empirical literature on informality. A 1\% increase in inflation around the steady state equilibrium reduces both informal employment and unemployment by 0.42\% and 0.03\% respectively. This corresponds to the interpretation that an increase in inflation affects the labor allocation from high separation rate informal jobs to low separation rate formal jobs which simultaneously reduces the size of the informal sector and reduces unemployment. An increase in nominal interest rates from 5\% to 18\% drives informality down from 35\% to around 20\% with a slightly negative impact on unemployment. However, increasing inflation by 1\% reduces the aggregate output by 0.04\%. This is because inflation taxes also formal monetary transactions which limits it's effectiveness as a tool against informality.

Increasing the availability of credit for formal firms by 1\% reduces unemployment and informal employment by 0.1\% and 0.33\% respectively. Interestingly enough, financial development is effective at reducing unemployment while inflation is better at fighting informality. This implies that a combination of the two instruments might be a better policy.

In line with the literature, an increase of 1\% in the payroll tax increases unemployment and informal employment by 0.48\% and 1.34\% respectively. Strengthening government monitoring or lowering red tape costs, two proxies for the quality of a country institutions such as the rule of law and the strength of the state, reduces both unemployment and informality without hurting output. However, we known that institutions are very sticky and any positive impact of improving them is not likely to show up over the short or medium run.

One would expect that a bigger informal sector strengthens the positive long run relation between inflation and unemployment. However, if we consider the differences in separation rates between formal and informal jobs then the long-run Philips curve might become vertical or slightly negatively sloped. This is because the composition of the labor force in terms of formal and informal jobs is affected by monetary conditions prevailing in the economy. An increase in inflation reduces the trade surplus for both formal and informal firms. This lowers the benefits of vacancy posting reducing in turn entry of firms. Lower firm entry pushes unemployment upwards which reduces the matching probability in the goods market further amplifying the fall in job creation and the rise of unemployment. This positive effect of inflation on unemployment is referred to as the matching channel. An opposite effect works through the hiring channel where an increase in inflation pushes new entrants towards formal jobs. Since formal jobs exhibit a lower job separation rate, this shift from informal to formal employment reduces unemployment. Since buyers are randomly matched to firms in the retail sector, the higher ratio of formal to informal firms increases the buyers' matching probability with formal firms which further reduces the attractiveness of informality. The existence of this two channels renders the effect of inflation on unemployment ambiguous. Preliminary numerical results show that, with a sizable informal sector, the hiring channel dominates the matching channel which means that inflation has a strong effect on labor reallocation between formal and informal jobs but a very small net effect on unemployment and output. These results point to the importance for central banks to take into consideration the size of the informal sector in the conduct of monetary policy and in particular with regard to the trade-off between inflation and unemployment.

\section{Macroeconomic Fluctuations and Informality}

My objective in the second project is to study the short-run relationship between inflation and unemployment in developing economies characterized by the presence of a large informal sector. The literature on the business cycle effects of informality is relatively small and does not consider the higher cash intensity of informal transactions compared to the formal economy which is fundamental in my opinion \citep{Conesa2002, Fiess2007, Castillo2008}.

This project will take the model presented in the first project as a starting point and introduce some modifications to improve its business cycle properties. As discussed in the first project, the base model draws heavily from the work of \cite{Berentsen2011} who present a framework that is very suitable for the study of the long-run interactions between money and the real economy. Their model is able to replicate the long-run positive correlation between inflation and unemployment. This performance is achieved through the Friedman channel which operates as follows: an increase in inflation increases the cost of holding money which induces households to reduce their money holdings. This in turn lowers the amount of goods exchanged in the KW market and reduces the firms surplus. As a consequence, firms tend to post less vacancies which increases unemployment. However, the short-run properties of their model are not very convincing. Indeed, the model is unable to replicate the short-run negative relationship between inflation and unemployment following an expansionary monetary policy shock which is observed in the data. Indeed, empirical evidence from the US shows that an expansionary monetary policy shock generates an increase in inflation and a decrease in unemployment in the short-run  \citep{Christiano2005}. Whether this is the case in developing economies remains to be seen since the literature on this issue is very limited. For this reason, we will conduct an empirical analysis of this issue by estimating a structural vector autoregressive model with short run restrictions to identify monetary shocks as in \cite{Christiano2005} using available data from some developing economies with a large informal sector. This will allow me to assess the empirical nature of the short-run relationship between inflation and unemployment in the presence of informality which we will then try to replicate through our extended model.

I plan to introduce four significant improvements. First, I will add both endogenous separation and on-the-job search in the labor market for both formal and informal workers. Second, I will introduce costs of labor regulation such as hiring and firing costs. Third, I will introduce wage stickiness in the form of staggered Nash wage bargaining in the formal employment market. Last, I will introduce some form of price rigidities. 

The first extension will be to allow workers to search on the job for other potential employment opportunities. For that I will introduce endogenous separation as in \cite{Mortensen1994} and on-the-job search following \cite{Pissarides1994}. This will allow for a better replication of job market flows at higher frequencies.

For the second extension, we plan to introduce fixed hiring and firing costs following \cite{Mortensen1999} to reflect the weight of labor regulation on the decision of firms and workers to operate under formal or informal work arrangements. In our case, hiring and firing costs are paid only by firms who choose to hire formal workers. Firing costs will also be incurred if a firm decides to change a formal work contract into an informal work arrangement. The introduction of fixed firing costs will distinguish between newly formed formal matches and ongoing formal matches. If a newly formed formal match is terminated, the firm is not subject to firing costs whereas it has to pay firing costs if an ongoing formal match is terminated. This will result in a two-tiered wage structure in the formal labor market as explained in \cite{Mortensen1999}. I expect this to improve the fit of the wage distribution of formal workers as well as offer the possibility to run experiments in terms of labor policies. This will also be very useful when calibrating the model to different economies which differ along their labor institutions.

The third extension has the potential of improving the business cycle performance of the model with respect to the wage fluctuations observed in the data. One of the problems with the standard MP model \citep{Mortensen1994} is that the period-by-period Nash wage bargaining generates highly volatile and pro-cyclical fluctuations in wages compared with the data. To address this issue, I plan to introduce some form of wage stickiness in the formal sector by modeling the bargaining mechanism in such a way that wages to not adjust automatically to changes in economic conditions. There are several mechanisms resulting in some form of wage stickiness such as staggered Nash bargaining as in \cite{Gertler2009} and strategic wage bargaining as in \cite{Hall2008}. A simpler mechanism presented by \cite{Gertler2009} extends the standard MP model by modifying the wage setting mechanism in such a way that wages remain fixed over a given horizon in a similar fashion to the price setting mechanism in \cite{Calvo1983}. For this to work, \cite{Gertler2009} assume that all workers in all matches with the same idiosyncratic productivity receive the same wage which is revised infrequently to reflect changes in aggregate productivity. Wage renegotiations take place with a probability following a Poisson process. Newly hired workers are paid the predominant wage level and their wages are negotiated only at the time when all wages are revised and not at recruitment. This results in the value of a new match depending on the currently prevailing wage as well as future anticipated wages as in the widely used Calvo pricing formula in \cite{Calvo1983}. In terms of the implied fluctuations, since wages do not react immediately following an aggregate productivity shock the response in the vacancy–unemployment ratio is large on impact but then gradually dissipates as wages are renegotiated and adjusted to the shock. It is worth noting that this wage setting mechanism generalizes both the standard case in which wages are continually revised and the case where wages are fixed. The authors introduce this wage setting mechanism in an environment where a firm employs a continuum of workers. However, this can be adapted to a one-firm-one-worker environment as in \cite{Mortensen2007}.

The last extension will be to introduce some kind of price stickiness into the model. Macroeconomic models with price stickiness are used heavily in the New Keynesian literature. This is because such models generate non-neutrality of money which allows monetary policy to play a role in stabilizing business cycle fluctuations. In models based on \cite{Lagos2005}, money plays an important role in the model thanks to the presence of search frictions in the KW market. As a consequence, there is no need to introduce nominal price rigidities in order to generate a non-neutrality of money. Nevertheless, we still need to introduce price stickiness because it is observed in the data. Indeed, not all sellers adjust their prices automatically following a change in the aggregate price level. One possibility would be to introduce a staggered price setting mechanism as in \cite{Calvo1983} to our model. This will need to transform the AD market from a competitive market where firms and households are price takers to a market of monopolistic competition where firms have some power in setting prices. This mechanism introduces price stickiness by allowing the response of firms to a shock to be spread out through time since not all firms are able to immediately adjust their prices. It results also in firms being forward looking in their pricing decision since they might not be able to change their prices in the future for some time to come.

I believe these modifications will improve the model's performance in terms of replicating business cycle stylized facts observed in developing economies. I plan to calibrate the model following \cite{Aruoba2011} and analyze the resulting impulse-response functions produced to assess changes in the reaction of unemployment to interest rate shocks as a function of the size of the informal sector.

\section{A Monetary Model of Financial Inclusion}

My third project is a joint work with Professor Pedro Gomis-Porqueras at Deakin University, Australia. In this work in progress we develop a theory of financial inclusion in emerging and developing economies.

Financial inclusion is broadly defined as the access to formal financial institutions offering instruments for payments, saving and borrowing for the purpose of consumption smoothing as well as long-term credit to finance durables consumption, housing and investment projects. The issue of financial inclusion has gained importance since the early 2000s, a result of robust empirical findings that associate financial exclusion with poverty.

While financial development relates to the ``depth" or intensity in the use of financial services, measured for example by private sector credit as a percentage of GDP, financial inclusion refers more to the ``breadth" of access of the population to basic financial services, measured for example by banking account ownership. According to data from the World Bank, 38\% of the global adult population did not own an account at a formal financial institution in 2014 \citep{Demirguec-Kunt2015}.

Among those who are financially excluded, the literature distinguishes between indirect users who have access to a bank account through another person, the self-excluded, usually for cultural or religious reasons, and involuntary excluded individuals. The most often cited reasons for involuntary financial exclusion are financial costs, distance, lack of information and lack of trust. According to \cite{Allen2016} higher financial inclusion is associated with lower costs of account ownership, lower physical costs of access to financial intermediaries, stronger legal rights and more political stability.

Changes in financial inclusion might interact with monetary policy in several dimensions. For example, an increase in the share of financially included households might affect the use of money and credit in transactions which might reflect through unanticipated changes in monetary aggregates. A central bank that tracks these monetary aggregates in the conduct of monetary policy should be able to distinguish between changes related to the level of the economic activity to those related to changes in financial inclusion. In addition, an increase in the access to financial services will improve the ability of consumers to smooth their consumption. This makes consumers more sensitive to changes in interest rates which might improve the potency of monetary policy. Changes in financial inclusion will hence induce changes in consumption and output volatility in response to monetary shocks. Also the level of financial inclusion in an economy determines how much this economy is affected by inflation which might matter for the unemployment-inflation trade-off in particular in the long run.

In this project, our objective is to develop a theory that sheds some light on the effects of monetary policy on banking competition and the access of consumers to basic financial services in economies with low levels of financial inclusion. For that, we build a model of money, banking and financial inclusion in an environment similar to \cite{Lagos2005} and \cite{Berentsen2007}. Agents exchange goods and services in various markets while financial intermediaries offer their financial services to agents. In each period, two competitive markets open in a consecutive fashion. These are characterized by different frictions. In the first market, anonymous agents trade a specialized good, which from now on we will refer as AM. In the second market, agents have access to a frictionless Walrasian market where a homogeneous good can be produced by all agents. This latter market is referred CM.

At the beginning of each period, agents are subject to two shocks. The first is a preference shock that determines whether an agent is a buyer or a seller in the AM. The second is a ``location" shock that determines how far away agents in the AM are located relative to the financial intermediaries. This ``location" shock  captures the physical and informational costs that agents face when accessing financial services during the first sub-period.\footnote{This type of costs has been emphasized by \cite{Allen2016} as being one of the main factors influencing access to  financial intermediaries.} In the second market, buyers and sellers can consume and produce a homogeneous perishable good.

Since agents are anonymous and have no access to record-keeping services in the first market, a medium of exchange is essential for trades to take place. On the other hand, since agents can produce and consume the CM, a medium of exchange in this market is not essential. The only durable asset in this economy is an intrinsically useless object, fiat money. As in \cite{Berentsen2007}, given that agents have access to a financial market before the specialized good trade takes place, financial intermediaries may accept one-period nominal deposits and offer one-period nominal loans. We refer to these intermediaries as banks as they perform some of a bank's functions. In order for financial intermediation to be possible, banks need to have access to a record keeping technology that allows banks to register the identity of agents making use of their services.

The use of a location shock allows us to endogenize the measure of financially included agents in a simple way. Agents whose location shocks are above a threshold will be willing to incur the cost of accessing the bank to borrow or deposit. In the current version of the model, the location thresholds for borrowers and depositors are different and depend on inflation and the nominal interest rates as well as on the degree of competition in the banking sector. We plan to explore the implications of the model under different competition structures in the banking sector: a perfectly competitive banking sector as well as imperfect banking competition in the spirit of the Monti-Klein model. We also intend to explore the effect of different monetary policy mechanisms such as lump-sum transfers to the financially included agents in order to promote financial inclusion as well as a discount window where banks can borrow fiat money from the central bank which they can later loan and pay back at the end of CM to the central bank, by modifying the amount of cash in their balance sheet then (depending on the competitive nature of banks) the lending and deposit rates will adjust. This makes it possible to decouple the amount of loans banks can provide from the amount of deposits they receive.


%----------------------------------------------------------------------------------------
%	BIBLIOGRAPHY
%----------------------------------------------------------------------------------------

\pagebreak

\label{Bibliography}

\addcontentsline{toc}{section}{References}

%\section*{Bibliography} % Change the page header to say "Bibliography"

\bibliographystyle{plainnat} % Use the "unsrtnat" BibTeX style for formatting the Bibliography

\bibliography{dissertation_proposal_ref} % The references (bibliography) information are stored in the file named "Bibliography.bib"

%\nocite{*}

\end{document}
